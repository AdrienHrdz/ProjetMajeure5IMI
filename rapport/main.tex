\documentclass{article}
\usepackage[utf8]{inputenc}

% Language setting
% Replace `english' with e.g. `spanish' to change the document language
\usepackage[french]{babel}

% Set page size and margins
% Replace `letterpaper' with `a4paper' for UK/EU standard size
\usepackage[letterpaper,top=2cm,bottom=2cm,left=3cm,right=3cm,marginparwidth=1.75cm]{geometry}

% Useful packages
\usepackage{amsmath}
\usepackage{graphicx}
\usepackage[colorlinks=true, allcolors=black]{hyperref}
\usepackage{float}
\usepackage{wrapfig}
\usepackage[T1]{fontenc}
\usepackage{comment}
\usepackage{authblk}
\usepackage{tikz}
\usepackage{pgfplots}
\pgfplotsset{compat=1.15}
\usepackage{mathrsfs}
\usetikzlibrary{arrows}
\usepackage{caption}
\usepackage{subcaption}
\usepackage{biblatex}
\addbibresource{Unity.bib}
\usepackage{csquotes}
\usepackage{multicol}






\title{Projet de fin de majeure}
\author{Adrien Hernandez - Vincent Merrouche}
\date{2022 / 2023}

\begin{document}

\maketitle

\section{Rappel du projet}

\section{Jalons}

\section{Organistation}

\subsection{Versionnement du code}
Nous avons fait le choix d'utiliser un service de versionnement du code afin de pouvoir bénéficier de sauvegardes et d'une simplicité pour travailler à plusieurs. Pour cela nous avons utliser le service GitHub basé sur le logiciel Git. 

Le repo distant était organisé sur 3 branches distinctes, une branche \textit{main}, et deux branches pour chacun des membres du groupe. Chaque membre travaillait donc sur sa branche et pouvait faire autant de commit que nécessaire. 

Nous avons fait le choix de ne rien commit sur la branche \textit{main} afin de ne pas avoir de conflits lors de la fusion des branches. Les commits sur la branche \textit{main} étaient donc effectués par \textit{pull request} afin de pouvoir vérifier les modifications apportées par les autres membres du groupe.

Nous avons eu du mal à mettre en place ce systèmes de branches et de pull request, mais nous avons finalement réussi à le mettre en place et à le faire fonctionner correctement. Ce type de fonctionnement est très présent dans le monde professionnel, c'était donc utile de se familiariser avec le principe.


\section{Résultats}

\end{document}