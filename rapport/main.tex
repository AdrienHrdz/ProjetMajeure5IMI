\documentclass{article}
\usepackage[utf8]{inputenc}

% Language setting
% Replace `english' with e.g. `spanish' to change the document language
\usepackage[french]{babel}

% Set page size and margins
% Replace `letterpaper' with `a4paper' for UK/EU standard size
\usepackage[letterpaper,top=2cm,bottom=2cm,left=3cm,right=3cm,marginparwidth=1.75cm]{geometry}

% Useful packages
\usepackage{amsmath}
\usepackage{graphicx}
\usepackage[colorlinks=true, allcolors=black]{hyperref}
\usepackage{float}
\usepackage{wrapfig}
\usepackage[T1]{fontenc}
\usepackage{comment}
\usepackage{authblk}
\usepackage{tikz}
\usepackage{pgfplots}
\pgfplotsset{compat=1.15}
\usepackage{mathrsfs}
\usetikzlibrary{arrows}
\usepackage{caption}
\usepackage{subcaption}
\usepackage{biblatex}
\addbibresource{Unity.bib}
\usepackage{csquotes}
\usepackage{multicol}






\title{Projet de fin de majeure}
\author{Adrien Hernandez - Vincent Merrouche}
\date{2022 / 2023}

\begin{document}

\maketitle

\section{Rappel du projet}
Dans le monde de la musique et des arts visuels, il existe une discipline appelée le V-Jing qui consiste à mixer des images et des sons en temps réel. 
Le but de ce projet était de fournir une application qui permettait de créer des animations visuelles basées sur un flux audio capté en temps réel.

\section{Jalons}
Initialement, nous avions prévu un ensemble de trois jalons à courts, moyens et long termes pour ce projet. Ces jalons étaient les suivants :
\begin{itemize}
    \item Jalon 1 : récupérer un flux audio, faire une transformée à court termes sur ce flux audio pour en obtenir le spectrogtamme, ainsi que de commencer des tests pour les animations.
    \item Jalon 2 : mise en commun des branches de développement, synthèse et traitement de signal; modifications des animations pour qu'elles soient impactées par le son.
    \item Jalon 3 : Complexfier les animations et faire de l'écoute temps réel.
\end{itemize}



\section{Organistation}
Pour être productif nous avons décidé de travailler à l'école afin de bénéficier d'une ambiance de travail et potentiellement de l'aide disponible sur place. A termes Vincent à décidé de travailler depuis chez lui car son ordinateur fixe est plus performant que son ordinateur portable et il dispose de 2 écrans. C'était donc plus pratique pour lui de travailler depuis chez lui. 

\section{Versionnement du code}
Nous avons fait le choix d'utiliser un service de versionnement du code afin de pouvoir bénéficier de sauvegardes et d'une simplicité pour travailler à plusieurs. Pour cela nous avons utliser le service GitHub basé sur le logiciel Git. 

Le repo distant était organisé sur 3 branches distinctes, une branche \textit{main}, et deux branches pour chacun des membres du groupe. Chaque membre travaillait donc sur sa branche et pouvait faire autant de commit que nécessaire. 

Nous avons fait le choix de ne rien commit sur la branche \textit{main} afin de ne pas avoir de conflits lors de la fusion des branches. Les commits sur la branche \textit{main} étaient donc effectués par \textit{pull request} afin de pouvoir vérifier les modifications apportées par les autres membres du groupe.

Nous avons eu du mal à mettre en place ce systèmes de branches et de pull request, mais nous avons finalement réussi à le mettre en place et à le faire fonctionner correctement. Ce type de fonctionnement est très présent dans le monde professionnel, c'était donc utile de se familiariser avec le principe.


\section{Résultats}

\subsection*{Jalon 1}
Le jalon 1 s'est bien passé. Il a été un peu retardé par la mise en place du versionnement. Nous avons eu quelques problèmes avec nos branches git respectives. 

\subsection*{Jalon 2}
Le jalon 2 s'est bien passé et les objectifs ont été atteints. De plus, la fusion des branches de développement git s'est passée sans problèmes majeurs. 

\subsection*{Jalon 3}
Le jalon 3 n'a pas été totalement complété. Nous avons réussi à produire des animations plus complexes, mais nous n'avons pas réussi à faire de l'écoute temps réel. L'écoute en temps réel s'est avérée plus compliquée que prévu, cependant avec un délai supplémentaire cette fonctionnalité aurait pu être implémentée.

\section{Conclusion}
Nous avons réussi à produire une application qui permet de visualiser des animations modifiées par la musique jouée. 
\end{document}

